\section{Main definitions.}

Let $G$ be a finitely generated (not necessarily abelian) group.
Let $\Omega$ be a metric space with a metric $\dist$. For $x \in \Omega$, $U \subset \Omega$, $\delta>0$ denote
$$
B(\delta, x) = \{y\in \Omega: \dist(x, y) < \delta\}, \quad
B(\delta, U) = \cup_{x \in U} B(\delta, x).
$$

We say that a map $\Phi: G \times \Omega \to \Omega$ is a (left) action of a group $G$ if the following holds:
\begin{itemize}
\item[(G1)] the map $f_g = \Phi(g, \cdot)$ is a homeomorphism of $\Omega$ for any $g \in G$;

\item[(G2)] $\Phi(e, x) \equiv x$, where $e \in G$ is the identity element of the group $G$;

\item[(G3)] $\Phi(g_1g_2, x) = \Phi(g_1, \Phi(g_2, x))$, for any $g_1, g_2 \in G$, $x \in \Omega$.
\end{itemize}

We say that an action $\Phi$ is \textit{uniformly continuous} if for some symmetric generating set $S$ (a generating set is called symmetric if together with any element $s \in S$ it contains $s^{-1}$) of a group $G$ the maps $f_{s}$ are uniformly continuous for all $s\in S$. Note that if $\Omega$ is compact, then any action of a finitely generated group is uniformly continuous.


Let us fix some finite symmetric generating set $S$ of a group $G$.

\begin{defin}\label{defPst}
For $d >0$ we say that a sequence $\{y_{g}\}_{g\in G}$ is a $d$-pseudo\-tra\-jec\-to\-ry of an action $\Phi$ (with respect to the generating set $S$) if
\begin{equation}
\label{dpstdef}
\dist(y_{sg}, f_s(y_g)) < d, \quad s \in S, g \in G.
\end{equation}
\end{defin}

\begin{defin}\label{defStSh}
We say that an uniformly continuous action $\Phi$ has the \textit{shadowing property} on a set $V \subset \Omega$ if for any $\ep > 0$ there exists $d > 0$ such that for any $d$-pseudotrajectory $\{y_g \in V\}_{g\in G}$ there exists a point $x_e \in \Omega$ such that
\begin{equation}\label{eqDefSh}
\dist(y_g, f_g(x_e)) < \ep, \quad g \in G.
\end{equation}
In this case we say that $\{y_g\}$ is $\ep$-shadowed by the exact trajectory $\{x_g = f_g(x_e)\}_{g\in G}$.
If $V = \Omega$ we simply say that $\Phi$ has the shadowing property.
\end{defin}

This notion is a natural generalization of the concept of the shadowing property introduced in \cite{PilTikh} for actions of $\ZZ^n$.

Let us also note that the definition of a pseudotrajectory depends on a choice of the generating set $S$. However the following proposition shows that if an  uniformly continuous action has shadowing for one finite symmetric generating set, then it has shadowing for any finite symmetric generating~set.

\begin{prp}
\label{prop}
Let $S$ and $S^{\prime}$ be two finite symmetric generating sets for a group $G$.  An uniformly continuous action $\Phi$ has the shadowing property on a set $V \subset \Omega$ with respect to the generating set $S$, if and only if it has the shadowing property on a set $V \subset \Omega$ with respect to the generating set $S^{\prime}$.
\end{prp}

The proof of Proposition \ref{prop} is straightforward, see Appendix.

The following notion of expansivity is important for our results:
\begin{defin}
An action $\Phi$ is \textit{expansive} (or has \textit{expansivity}) on a set $U \subset \Omega$ if there exists $\Delta > 0$ such that if
$$
\Phi(g, x_1),\Phi(g, x_2) \in U, \quad \dist(\Phi(g, x_1), \Phi(g, x_2)) < \Delta, \quad \forall g \in G
$$
for some $x_1, x_2 \in U$, then
 $x_1 = x_2$.
\end{defin}
Note that if $G_1 \leq G$ is a subgroup of $G$ and $\Phi|_{G_1}$ has expansivity, then~$\Phi$ has expansivity too.

Any homeomorphism $f: \Omega \to \Omega$ induces an action $\Phi_f: \mathbb{Z} \times \Omega \to \Omega$ of the group $\mathbb{Z}$ defined as $\Phi_f(k, x) = f^k(x)$ for $k\in\mathbb{Z}$.
 We say that
\begin{enumerate}
\item a %uniformly continuous
homeomorphism $f$ has shadowing on a set \hbox{$V \subset \Omega$};
\item a homeomorphism $f$ has expansivity on a set $U \subset \Omega$;
\end{enumerate}
if the corresponding action $\Phi_f$ has this properties. Note that these definitions are equivalent to classical definitions of these notions.

\begin{defin}
Consider two sets $U, V \subset \Omega$. We say that an uniformly continuous action $\Phi$ is topologically Anosov with respect to the pair $(U, V)$ if the following conditions are satisfied:
\begin{itemize}
\item[(TA1)] there exists $\gamma > 0$ such that $B(\gamma, V) \subset U$;
\item[(TA2)] $\Phi$ has the shadowing property on $V$;
\item[(TA3)] $\Phi$ is expansive on $U$.
\end{itemize}
\end{defin}

