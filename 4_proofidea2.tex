\begin{deffnon}
We say that $f$ satisfy the \indef{analytic transversality condition} at a point $p$ if
\begin{equation*}
B^+(p)+B^-(p)=T_pM.
\end{equation*}
\end{deffnon}

\begin{theoremnon}[Ma\~n\'e, \cite{MANECASD}] Diffeomorphism $f$ is structurally stable iff it satisfies analytic transversality condition for any $p\in M$.
\end{theoremnon}

So, to get structural stability one need to provide that differentials of $f^k$ satisfy some linear algebraic condition. On the other hand it could be shown that the existence of the Lipschitz shadowing implies the solvability of the linear system of difference equations for the sequence $A_k=Df^k(p).$ In fact it provides the existence of a bounded solution of \eqref{eq:nonhomogen} for every bounded inhomogeneity.  This solvability is also some linear algebraic condition. The Pliss' theorem gives connection between this two conditions.
