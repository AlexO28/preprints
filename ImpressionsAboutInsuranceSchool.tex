\documentclass{beamer}
\usepackage{graphicx}
\usetheme{Berlin}
\title{Impressions about School on Insurance, Finance, and Risk Management.}
\author{Alexey Osipov}
\begin{document}
\maketitle
\begin{frame}[fragile]
\frametitle{Presentation 2.}
\begin{center}
\Large{Tail risk measures}
\end{center}
\end{frame}
\begin{frame}[fragile]
\frametitle{Risk measures.}
Suppose we have data on losses. How to measure risk of this losses?

\begin{itemize}
	\item \textbf{mean} (average loss)
	\item \textbf{variance} (average deviation from the average loss)
	\item \textbf{variance premium} (combination of the 2 measures above)
	$$VP(X) = E(X) + \alpha Var(X),$$
	where $\alpha$ is a smartly chosen constant (analog of loading?).
	\item \textbf{VaR} (how much money is required to cover 95\% of losses for sure)
\end{itemize}
\end{frame}
\begin{frame}[fragile]
\frametitle{Tail risk measures.}
What if we are interested mostly in rare events (5\% of worst cases)?

\begin{itemize}
	\item TE(X), \textbf{tail conditional expectation} (shortfall, \textbf{TVaR}),
	\item TV(X), \textbf{tail variance}
	\item TVP(X), \textbf{tail variance premium}
		$$TVP(X) = TE(X) + \alpha TV(X).$$
\end{itemize}
\end{frame}
\begin{frame}[fragile]
\frametitle{Elliptical distributions.}
Elliptical (mostly for finance):
\begin{figure}[h!]
\centering
\includegraphics[width=50mm, scale=0.1]{EllipticalDistributions.jpg}
\end{figure}
Nonelliptical: (multivariate) Pareto (mostly for insurance)
\end{frame}
\begin{frame}[fragile]
\frametitle{Asymptotic behaviour in one-dimensional case.}
Z. Landsman has results for asymptotic behaviour of $TE_q(X)$ and $TV_q(X)$ (when $q$ approaches 1) for:
\begin{itemize}
	\item one-dimensional normal distribution
	\item one-dimensional Student distribution
\end{itemize}
\end{frame}
\begin{frame}[fragile]
\frametitle{Multivariate tail risk measures.}
\begin{itemize}
	\item MTE (multivariate tail conditional expectation)
	\item MTCOV (multivariate tail covariance)
	\item MTCOR (multivariate tail correlation)
\end{itemize}
Z. Landsman has more or less explicit formulas for this tail risk measures for elliptical distributions. 
\end{frame}
\begin{frame}[fragile]
\frametitle{Risk measures and portfolio management.}
We have a number of assets/stocks: $X_1, \ldots, X_n$. Our goal is to construct optimal portfolio out of this assets:
$$X = w_1X_1 + \ldots + w_nX_n,$$
$$\sum_{i=1}^n w_i = 1.$$
so that certain risk measure is optimized:
$$\rho(X) \longrightarrow \min.$$
%Classical situation is:
%$$\rho(X) = E(X) + \alpha*Var(X)$$
%(classical mean-variance optimization).

Z. Landsman has results for portfolio optimization for various risk measures (all the risk measures discussed in the talk).
\end{frame}
\begin{frame}[fragile]
\frametitle{Presentation 1.}
\begin{center}
\Large{Claims evaluation}
\end{center}
\end{frame}
\begin{frame}[fragile]
\frametitle{Claims: goal and examples.}
\begin{itemize}
  \item We want to estimate the current fair values of the future claims (e.g., in 1 year).
	\item $X$ is euro-dollar exchange rate in a year
	\item $Y$ is 1 euro if insured dies within 1 year, otherwise it is 0
	\item $X\cdot Y$ is the insured contract in dollars.
\end{itemize}
\end{frame}
\begin{frame}[fragile]
\frametitle{Evaluation: questions.}
\begin{enumerate}
\item \textbf{Financial quant.} What is a price of claim when traded in an arbitrage-free market?
\item \textbf{Traditional actuary.} What is a price for taking over the liability ignoring the financial market except the risk-free bank account?
\item \textbf{Modern actuary.} What is a price for taking over the liability taking into account hedging opportunities in the financial market?
\end{enumerate}
\end{frame}
\begin{frame}[fragile]
\frametitle{Non-arbitrage market.}
\begin{itemize}
  \item we have $n+1$ assets, we want to connect there values at different time moments
	\item $r$ is the risk free rate, time period is 1 (e.g., year)
	\item the assets are liquid
	\item the assets are not-redundant
	\item the market is \textit{arbitrage-free}: starting from 0 we can not make some profit with some probability with probability 1 of not losing
\end{itemize}
\end{frame}
\begin{frame}[fragile]
\frametitle{Financial valuation.}
\begin{itemize}
\item Suppose that market is arbitrage-free.
\item \textbf{Hedgeable claim} is a claim that is a result of some trading strategy involving the assets in the market.
\item \textbf{Hedgeable claim:} price of euro-dollar in a year.
\item \textbf{Non-hedgeable claim:} price of euro-dollar now, life insurance contract.
\item Hedgeable claims can be evaluated using techniques from financial mathematics (EMM-measures).
$$value(S) = e^{-r}E^Q(S).$$
\item But non hedgeable claims cannot be estimated this way.
\end{itemize}
\end{frame}
\begin{frame}[fragile]
\frametitle{Actuarial valuation.}
\begin{itemize}
\item We have a market of $n$ traded assets/claims.
\item \textbf{Orthogonal claim} is a claim that is independent of the traded assets.
\item \textbf{Orthogonal claim:}

$Y$ is 1 euro if insured dies within 1 year, otherwise it is 0
\item \textbf{Non-orthogonal claim:} euro-dollar rate in a year.
\item Orthogonal claims can be evaluated using techniques from actuarial mathematics (actuarial model, risk margin)
$$value(S) = e^{-r}(E^P(S) + RM(S)).$$
\end{itemize}
\end{frame}
\begin{frame}[fragile]
\frametitle{Examples of risk margins.}
\begin{itemize}
\item $r$ is the risk-free rate.
\item Cost-of-capital principle:
$$RM(S) = e^{-r}i(VaR^P_{\alpha}(S) - E^P(S)),$$
where $i$ is the cost-of-capital rate.
\item Standard deviation principle:
$$RM(S) = e^{-r}(E^P(S) + \alpha \sigma^P(S)).$$
\item Or just
$$RM(S) = 2e^{-r}\sigma^P(S).$$
\end{itemize}
\end{frame}
\begin{frame}[fragile]
\frametitle{Valuation of hybrid claims.}
Karim Barigou and Jan Dhaene offer a way of evaluating some types of \textbf{hybrid claims}. This claims have both hedgeable and orthogonal parts. It is the combination of financial and actuarial evaluations.

Typical examples are:
\begin{itemize}
\item \textbf{Hedgeable claim:} $X$ is euro-dollar exchange rate in a year
\item \textbf{Orthogonal claim:} $Y$ is 1 euro if insured dies within 1 year, otherwise it is 0
\item \textbf{Hybrid claim:} $X\cdot Y$ is the insured contract in dollars.
\end{itemize}
$$S = S^h\times S^o,$$
$$value(S) = e^{-r} E^Q(S^h)\times E^P(S^o).$$
Case of several years is also considered.
\end{frame}
\begin{frame}[fragile]
\frametitle{Presentation 3.}
\begin{center}
\Large{Options for dependent assets.}
\end{center}
\end{frame}
\begin{frame}[fragile]
\frametitle{Problems from financial mathematics}
\begin{enumerate}
	\item \textbf{Portfolio selection.} Which stocks will go up, which will go down?
	
	\begin{itemize}
		\item 	Individual stocks are studied separately.
    \item Volatility and average return.
		\item Dynamics of the stock price process.
	\end{itemize}
	\item \textbf{Systemic risk measurement.} How likely are the stocks to move (down) together?
	
\begin{itemize}
	\item 	Dependence between stocks is studied.
\item Multivariate modeling.
\item Copulas.
\end{itemize}
\end{enumerate}
\end{frame}
\begin{frame}[fragile]
\frametitle{Estimating mean and variance.}
\begin{figure}[h!]
\centering
\includegraphics[width=50mm, scale=0.1]{rolling_indicators.jpg}
\end{figure}
Estimating mean is much more difficult than estimating variance in the word of finance.
\end{frame}
\begin{frame}[fragile]
\frametitle{Options.}
\begin{enumerate}
	\item A derivative is an asset which value depends on values of other more basic/underlying assets.
	\item \textbf{European call option.} The \textbf{right} to buy asset $S$ by strike price $K$ at date $T$. 
	\item \textbf{European put option.} The \textbf{right} to sell asset $S$ by strike price $K$ at date $T$.
	\item \textbf{American options} are almost the same. Not at date $T$ but no later than date $T$.
	\item What if $S$ is a basket? That is a combination of stocks, like an index...
\end{enumerate}
\end{frame}
\begin{frame}
\frametitle{Black-Scholes model.}
\begin{enumerate}
\item Put-call parity.
$$C(K) = P(K) + S - e^{-rT}K.$$
\item Black-Scholes equation.

The option price is a solution of PDE (with boundary conditions depending on the option type)

$$\frac{\partial V}{\partial t} + \frac{1}{2}\sigma^2S^2\frac{\partial^2 V}{\partial S^2}
+ rS\frac{\partial V}{\partial S} - rV=0$$

\item Black-Scholes model.

$$C(K) = N(d_1)S - N(d_2)e^{-rT}K$$
for suitable $d_1$ and $d_2$.
\end{enumerate}
\end{frame}
\begin{frame}
\frametitle{Options on indices.}
\begin{enumerate}
	\item Index is a weighted sum of stocks:
	$$I = w_1S_1 + \ldots + w_n S_n.$$
	\item Indices are not traded.
	\item Stocks are dependent
	\item Either approximate the option price.
	\item Or derive the upper bound for it.
\end{enumerate}
\end{frame}
\end{document}