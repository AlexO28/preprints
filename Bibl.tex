\begin{thebibliography}{99}
\bibitem{Anosov} D.V. Anosov, On a class of invariant sets for smooth dynamical systems, Proc. Fifth Internat. Conf. on Nonlinear Oscillations, vol. 2, Math. Inst. Ukranian Acad. Sci., Kiev, 1970, pp. 39--45. (in Russian)
\bibitem{Barbot} T. Barbot, C. Maquera, Algebraic Anosov actions of nilpotent Lie groups,
Topol. and its Applic., vol. 160, N. 1, 2013, pp. 199--219.
%, 2012, preprint, arxiv.org/abs/1207.0325v1.
    \bibitem{Bech} H. Bechtell, The theory of groups, Addison-Wesley Publishing Co., Reading, Mass.-London, Don Mills, Ont., 1971.
    \bibitem{Bowen} R. Bowen, Equilibrium states and the ergodic theory of Anosov diffeomorphisms, Lecture Notes in Math., Vol. 470, Springer-Verlag, Berlin-New York, 1975.
    \bibitem{BrHaef} M. R. Bridson, A. Haefliger, Metric spaces of non-positive curvature, Grundlehren der math. Wiss. 319, Springer-Verlag, Berlin, 1999.
%            \bibitem{Katok} D. Damjanovic, A. Katok, Local rigidity of partially hyperbolic actions, I. KAM method and $\mathbb{Z}^k$ actions on the torus, Second series, Vol. 172, N 3, 2010.
            \bibitem{Katok2} R. Feres, A.B. Katok, Ergodic theory and dynamics of $G$-spaces (with special emphasis on rigidity phenomena), Handbook of dynamical systems, 1A, edited by B. Hasselblatt and A. Katok, pp. 665--763, Elsevier, Amsterdam, 2002. %http://www.math.wustl.edu/\~feres/publications.html.
           \bibitem{Fisher} D. Fisher, Local rigidity of group actions: past, present, future, Recent Progr. in Dynamics MSRI Publications, Vol. 54, 2007.
          \bibitem{Gromov}
    M. Gromov, Groups of polynomial growth and expanding maps, Publications Mathematiques de I'IHES, 53, 1981, pp. 53--78.
%        \bibitem{Gromov0}
%    M. Gromov, Hyperbolic groups, Essays in group theory, Math. Sci. Res. Inst. Publ., 8, Springer, New York, 1987, pp. 75--283.
\bibitem{Hurder} S. Hurder, A survey of rigidity theory for Anosov actions, in ''Differential topology, foliations, and group actions'' (Rio de Janeiro, 1992), Contemp. Math., vol. 161, Amer. Math. Soc., Providence, RI, 1994, pp. 143--173.
        \bibitem{PdH} P. de la Harpe, Topics in geometric group theory, Chicago University press, 2000.
  %      \bibitem{Katok3} A.B. Katok, R.J. Spatzier, Differential rigidity of Anosov actions of higher rank abelian groups and algebraic lattice actions, Tr. Mat. Inst. Steklova, Vol. 216, 1997, pp. 292--319.
    \bibitem{Kur} A.G. Kurosh, The theory of groups, AMS, 2003.
      \bibitem{PilBook} S. Yu. Pilyugin, Shadowing in Dynamical Systems., Lecture notes math., vol. 1706, Springer, Berlin, 1999.
      \bibitem{Pil2} S. Yu. Pilyugin, Theory of pseudo-orbit shadowing in dynamical systems, Diff. Eqs, 2011, Vol. 47, N. 13, pp. 1929--1938.
          \bibitem{PilTikh}  S. Yu. Pilyugin, S. B. Tikhomirov, Shadowing in actions of some abelian groups. Fund. Math., 2003, Vol. 179, pp. 83--96.
    \bibitem{PalmBook} K.J. Palmer, Shadowing in dynamical systems. Theory and Applications, Kluwer, Dordrecht, 2000.
        \bibitem{Seg} D. Segal, Polycyclic groups, Cambr. Univ. Press, 1985.
       %\bibitem{St2} J.R. Stallings, Group theory and 3-dimensional %manifolds, Yale Math. monographs, Vol. 4, Yale university press, New %Haven, 1971.
    %\bibitem{St1} J.R. Stallings, On torsion-free groups with infinitely many %ends, Ann. of Math. 88, 1968, pp. 312 -- 334.
%        \bibitem{SupHi} D.A. Suprunenko, K.A. Hirsch, Matrix groups, 1976, %AMS.
%    \bibitem{Semmes} S. Semmes, An introduction to Heisenberg groups in analysis and geometry, Notices of the AMS, Vol. 50, N 6, 2003, pp. 640--646.
%%%%%%%%%%%%%%%%%%%%%
%    \bibitem{Runde} V. Runde, Lectures on amenability, Lecture notes in math., Springer, 2002.
%    \bibitem{Mil} J. Milnor, Growth of finitely generated solvable groups, J. Diff. Geom., 2, 1968, pp. 447--449.
%    \bibitem{Grig} R.I. Grigorchuk, On Burnside's problem on periodic groups. (Russian), Fun. Anal and ego pril., vol. 14, 1980, pp. 53--54.
%     \bibitem{PilTikhLipSh} Pilyugin S. Yu., Tikhomirov S. Lipschitz Shadowing implies structural stability.
\end{thebibliography}
