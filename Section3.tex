\section{Finitely generated groups.}

In this section we will outline basic notions from theory of finitely generated groups,
give relevant definitions, and formulate statements that we use in the
sequel. We refer the interested reader to the following books on group theory:~\cite{Bech, BrHaef, PdH, Kur}.

A group $G$ is called \textit{abelian} if $[g,h]:=ghg^{-1}h^{-1}=e$ for any $g,h\in G$.

\begin{defin}
Any abelian group is called a \textit{nilpotent group of class 0}.
A group $G$ is called \textit{nilpotent of class $n$} if it has the \textit{lower central series} of length~$n$:
\begin{equation}\notag
\label{nilpdef}
G=G_1\rhd \ldots\rhd G_{n+1}=e,\quad\mbox{where }G_{i+1}=[G_i,G],
\end{equation}
(as usual, $G_i\rhd G_{i+1}$ means that $G_{i+1}$ is a normal subgroup of $G_i$).
\end{defin}

The simplest nontrivial example of a nilpotent group is a so-called \textit{Heisenberg group} (see \cite{PdH}, \cite{Kur}): $<a,b,c\mid c=[a,b], ac=ca,bc=cb>$.

\begin{defin}
A group is called \textit{virtually nilpotent} if it has a nilpotent normal subgroup of a finite index (i.e. the corresponding factor group is finite).
\end{defin}

\begin{remark}\label{remFinGen}
Note that any subgroup of a finitely generated virtually nilpotent group is finitely generated. In fact the similar statement holds for a more general class of polycyclic groups (see \cite{BrHaef, Seg} for the details).
\end{remark}

Virtually nilpotent groups are important due to the celebrated theorem of Gromov: Any group of polynomial growth is virtually nilpotent. We refer the reader to \cite{Gromov} for the precise statement.


\begin{defin}
A group is called \textit{solvable} or \textit{soluble} if there exists a \textit{subnormal series} (of not necessarily finitely generated groups) $e=G_n\lhd\ldots\lhd G_1\lhd G_0=G$ such that $G_i/G_{i+1}$ is an abelian group. %This series are called \textit{subnormal} series.
\end{defin}

We study Baumslag-Solitar groups (see \cite{PdH}):
$$
BS(m,n)=<a,b\mid ab^ma^{-1}=b^n>,\quad m,n\in\mathbb{Z},
$$
which are solvable for $m=1$. These groups are well known in group theory as a source of numerous counterexamples.

We study actions of $F_n=<a_1,\ldots,a_n\mid\cdot>$, the \textit{free group} with $n$ generators, which is obviously not solvable.

