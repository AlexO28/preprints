%\documentclass[final,notheorems,hyperref={pdfpagelabels=false}]{beamer}
\documentclass[final,notheorems]{beamer}

\mode<presentation> {
  \usetheme{Tau}
}

%\usepackage[orientation=portrait,size=a0,scale=1.4,debug]{beamerposter}
\usepackage[orientation=portrait,size=a0,scale=1.4,debug]{beamerposter}

%\usefonttheme{serif}
\usefonttheme{professionalfonts}

\usepackage[english]{babel}
\usepackage{multirow}
\usepackage{amsmath}
\usepackage{amsfonts,amssymb}
\usepackage[retainorgcmds]{IEEEtrantools}
\usepackage{verbatim}

\usepackage{setspace}

\definecolor{defwordcolor}{rgb}{0.7,0.0,0.0}
\setbeamercolor{alerted text}{fg=defwordcolor}

%%%%%%%%%%%%%%%%%%%%%%%%%%%%%%%%%%%%%%%
%%%%%%%%%%%%%%%%%%%%%%%%%%%%%%%%%%%%%%%
%%%%%%%%%%%%%%%%%%%%%%%%%%%%%%%%%%%%%%%
%%%%%%%%%%%%%%%%%%%%%%%%%%%%%%%%%%%%%%%

\newcommand{\Rb}{\mathbb R}
\newcommand{\Nb}{\mathbb N}
\newcommand{\Zb}{\mathbb Z}
\newcommand{\Lc}{\mathcal L}
\newcommand{\eps}{\varepsilon}
\newcommand{\dx}{\textbf{dx}}
\newcommand{\lb}{\left(}
\newcommand{\rb}{\right)}
\newcommand{\Ac}{\mathcal{A}}
\newcommand{\diff}[2]{\frac{\partial}{\partial #2}#1}
\newcommand{\diffd}[2]{\frac{\partial^2}{\partial #1 \partial #2}}
\newcommand{\sumd}[2]{\sum_{u = #1}^{#2}}
\newcommand{\ceilr}{\lceil r \rceil}
\newcommand{\floor}[1]{\lfloor #1 \rfloor}
\newcommand{\enbrace}[1]{\lb #1 \rb}
\newcommand{\inv}[1]{{#1}^{-1}}
\newcommand{\setdef}[2]{\left\{ #1\ \left|\ #2 \right.\right\}}
\newcommand{\seqz}[2]{\left\{ {#1}_{#2} \right\}_{#2\in\Zb}}
\newcommand{\seq}[1]{ \{ #1 \} }

\newcommand{\half}[1]{\frac{#1}{2}}

\newcommand{\vmod}[1]{\left| #1 \right|}

\newcommand{\Zplus}{\Zb^+}
\newcommand{\Zminus}{\Zb^-}
\newcommand{\Rplus}{\Rb^+}
\newcommand{\Rminus}{\Rb^-}

\newcommand{\Nc}{\mathcal{N}}
\newcommand{\normB}[1]{\left\| #1 \right\|_\infty}

\newcommand{\normop}[1]{\left\| #1 \right\|}
\newcommand{\normban}[1]{\left\| #1 \right\|}

\newcommand{\sfm}[2]{X_{#1} P X_{-#2} }
\newcommand{\ufm}[2]{X_{#1} \enbrace{I-P} X_{-#2} }

\newcommand{\ek}{\exp_{p_k}}
\newcommand{\eki}{\exp_{p_k}^{-1}}
\newcommand{\emk}{\exp^{-1}_{p_{k+1}}}
\newcommand{\ekk}{\exp_{p_{k+1}}}
\newcommand{\ekki}{\exp^{-1}_{p_{k+1}}}
\newcommand{\Nint}{[-N,N-1]} %for eqs
\newcommand{\Nintw}{[-N+1,N]} %for inhoms
\newcommand{\Nints}{[-N,N]} %for solutions

\newcommand{\rayp}{{\Zplus}}
\newcommand{\rayn}{{\Zminus}}

\newcommand{\imply}{$\Rightarrow\ $}

\newcommand{\indef}[1]{\alert{{#1}}}
%\texttt

%\newcommand{\devskip}{\vskip-60pt}
%\newcommand{\devskipcol}{\vskip-35pt}
%\newcommand{\devskipcols}{\vskip-30pt}

\newcommand{\devskip}{\vskip-60pt}
\newcommand{\devskipcol}{}
\newcommand{\devskipcols}{}


\theoremstyle{plain} \newtheorem{theorem}{Theorem}
\theoremstyle{plain} \newtheorem*{theoremnon}{Theorem}
\theoremstyle{definition} \newtheorem{deff}{Definition}
\theoremstyle{definition} \newtheorem*{deffnon}{Definition}
\theoremstyle{remark} \newtheorem{rem}{Remark}

\makeatletter
\setbeamertemplate{theorem begin}
{%
\vspace{0.5em}
\slshape %ignore body font
%\begin{\inserttheoremblockenv}
{%
\inserttheoremheadfont
\inserttheoremname
\inserttheoremnumber
\ifx\inserttheoremaddition\@empty
\else\ (\inserttheoremaddition)\fi%
\inserttheorempunctuation\hskip20pt
}%
}
\setbeamertemplate{theorem end}
{
%\end{\inserttheoremblockenv}
\vspace{0.5em}
}
\makeatletter

\DeclareMathOperator{\dist}{dist}


\institute[Chebyshev laboratory]{Chebyshev laboratory, Department of Mathematics and Mechanics, Saint Petersburg State University, Saint Petersburg, Russia}

\subject{ICTP 2012}

\title{\huge Analogues  of  Theorems  of  Maizel  And  Pliss  And  Their  Applications  in  The  Shadowing  Theory}
\author[Dmitry Todorov]{Dmitry Todorov \\ todorovdi@gmail.com}
\mail{todorovdi@gmail.com}


\newlength{\columnheight}
\setlength{\columnheight}{105cm}
%%%%%%%%%%%%%%
%%%%%%%%%%%%%%
\begin{document}

\begin{frame}
\devskipcols
\begin{columns}

% Col1
\begin{column}{.325\textwidth}
%
\parbox[t][\columnheight]{\textwidth}{
%
\devskipcol
%
%
\begin{block}{The Concept of Shadowing}
\begin{minipage}{0.97\textwidth}
%
\setlength{\parindent}{1em}	
%
\input{1_intro.tex}
%
\end{minipage}
\end{block}
%
\begin{block}{Classical Shadowing}
\begin{minipage}{0.97\textwidth}
%
Let $f$ be a diffeomorphism of a Riemannian manifold $M.$
%
\begin{deffnon}
A sequence $\seqz{x}{k}$ of points of $M$ is a \indef{$d$-pseudotrajectory} if
\begin{equation*}
\dist(x_{k+1},f(x_k))<d,\quad k\in\Zb.
\end{equation*}
\end{deffnon}
%
\devskip
%
\begin{deffnon}
$f$ has \indef{shadowing property} (pseudo-orbit tracing property, $f\in POTP (=StSh)$) if for a given $\eps$ there exists $d > 0$ such that for any $d$-pseudotrajectory $\{x_k\}$ there exists a point $p\in M,$ such that
$$
\dist(x_{k},f^k(p)) \leq  \eps,\quad k\in\Zb.
$$
\end{deffnon}
%
We denote the set of all structurally stable diffeomorphisms of $M$ by $SS.$
\begin{theoremnon}[Sakai, 1994]
$$SS=Int^1 (POTP).$$
\end{theoremnon} 
%
\devskip
%
\end{minipage}
\end{block}
%
%
\begin{block}{Limit Shadowing}
\begin{minipage}{0.97\textwidth}
%
\input{LimitShadowing.tex}
%
%
%
\end{minipage}
\end{block}
%
%
\begin{block}{Lipschitz Shadowing}
\begin{minipage}{0.97\textwidth}
%
\begin{deffnon}
$f$ has \indef{Lipschitz shadowing property} ($f\in LSP$) if there exist constants
$L, d_0 > 0$ with the following property: for any $d$-pseudotrajectory $\{x_k\}$ with $d < d_0$ there exists a point $p\in M,$ such that
$$
\dist(x_{k},f^k(p)) \leq  Ld,\quad k\in\Zb.
$$
\end{deffnon}
%
Lipschitz shadowing gives Lipschitz control on accuracy.
%
\begin{theoremnon}[Pilyugin, Tikhomirov, 2010] \label{thm:mainlip}
\alert{$$SS=LSP.$$}
%A diffeomorphism $f$ is structurally stable iff $f\in LSP.$
\end{theoremnon} 
%
\devskip
\begin{theoremnon}[Fakhari, Lee, Tajbakhsh, 2011]
Let $p\geq 1.$ Then
$$ L^p SP \subset SS,$$
$$ Int^1(L^p SP) = SS. $$
\end{theoremnon}
%
\end{minipage}
\end{block}
%
%



%\vfill

%\begin{block}{References}
%\begin{minipage}{0.97\textwidth}
%\small
%\vskip-1em
%%\begin{spacing}{0.8}
%\input{8_references.tex}
%%\end{spacing}
%\end{minipage}
%\end{block}

}



\end{column}


% Col2
\begin{column}{.325\textwidth}
%
%
\parbox[t][\columnheight]{\textwidth}{
%
\devskipcol
%
\begin{block}{LSP \imply SS}
\begin{minipage}{0.97\textwidth}
Fix $p\in M.$ Let $A_k = Df( f^k(p) ).$

\begin{enumerate}
    \item \label{item:shadimplsolv}  Shadowing \imply solvability of system of linear difference equations
    $$
    x_{k+1} = A_k x_k + f_{k+1},\qquad k\in \Zb.
      $$
      Bounded $\seqz{f}{k}$ \imply bounded solution $\seqz{x}{k}$
    \pause

    \item \label{item:solvimplhyp} Solvability \imply special case of non-uniform hyperbolicity \pause
    \item \label{item:hypimplstab} Special case of non-uniform hyperbolicity \imply structural stability
\end{enumerate}
%

\input{3_shadandstab2.tex}
\end{minipage}
\end{block}
%
\begin{block}{Step \ref{item:shadimplsolv}. Shadowing \imply solvability}
\begin{minipage}{0.97\textwidth}
\begin{itemize}
\item Take inhomogeneity $\seqz{w}{k}$ \pause
\item Select small $d$ and make small shifts of points $f^k(p)$ of some exact trajectory by $dw_k$ \pause
\item Shadow this $d$-pseudotrajectory \pause
\item If $d$ was small enough, you're done!
\end{itemize} 
\end{minipage}
\end{block}
%
\begin{block}{Step \ref{item:hypimplstab}. Shadowing \imply solvability}
\begin{minipage}{0.97\textwidth}
One of the ways to obtain the structural stability is to use Ma\~n\'e theorem.	

For a point $p\in M$ consider two linear subspaces of $T_pM:$
\begin{IEEEeqnarray*}{rcl}
B^+(p)&=& \setdef{v\in T_pM}{\vmod{Df^k(p)v}\to 0,\quad k\to +\infty},\\
B^-(p)&=&\setdef{v\in T_pM}{\vmod{Df^k(p)v}\to 0,\quad k\to -\infty}.
\end{IEEEeqnarray*}

\begin{deffnon}
We say that $f$ satisfy the \indef{analytic transversality condition} at a point $p$ if
\begin{equation*}
B^+(p)+B^-(p)=T_pM.
\end{equation*}
\end{deffnon}

\begin{theoremnon}[Ma\~n\'e, 1977] Diffeomorphism $f$ is structurally stable iff it satisfies analytic transversality condition for any $p\in M$.
\end{theoremnon}
\end{minipage}
\end{block}
%
\begin{block}{Step \ref{item:solvimplhyp}. Hyperbolicity \imply stability}
\begin{minipage}{0.97\textwidth}
There is a relation between the solvability of the system of linear difference equations and some hyperbolicity properties of this system.
\newline
Let $I$ be either $\Zplus = \setdef{k\in\Zb}{k\geq 0}$ or $\Zminus = \setdef{k\in\Zb}{k\leq 0}$ or $\Zb.$

Let $\Ac=\{A_k\}_{k\in I}$ be a sequence of linear isomorphisms $\Rb^d\to\Rb^d$ indexed by integers from $I.$ Consider homogeneous and inhomogeneous equations associated with this sequence.

\begin{IEEEeqnarray*}{rCll}
x_{k+1}&=&A_k x_k,\qquad &k\in I; \label{eq:homogen} \\
x_{k+1}&=&A_k x_k + f_{k+1},\qquad &k\in I. \label{eq:nonhomogen}
\end{IEEEeqnarray*}

Fix $\omega \geq 0.$ Denote the Banach space of sequences with bounded norm $\normban{x}_\omega=\sup\limits_{k\in I} \vmod{x_k} (|k|+1)^\omega$ by $\Nc_{\omega}(I).$ Such spaces have been already studied in a similar context (Bichekguev, 2011).

\begin{deffnon}
A sequence $\Ac$ has \indef{Perron property $B_{\omega}(I)$ } if $\seqz{f}{k}\in \Nc_\omega(I)$ \imply solution of inhomogeneous equation $\seqz{x}{k} \in \Nc_\omega(I).$
\end{deffnon}

Such properties are often called \indef{admissibility.}



\end{minipage}
\end{block}
%
%
%
%\begin{block}{$L^p$ Shadowing}
%\begin{minipage}{0.97\textwidth}
%%
%\begin{theoremnon}[Fakhari, Lee, Tajbakhsh, 2011]
Let $p\geq 1.$ Then
$$ L^p SP \subset SS,$$
$$ Int^1(L^p SP) = SS. $$
\end{theoremnon}
%%
%\end{minipage}
%\end{block}
}
\end{column}


% Col3
\begin{column}{.325\textwidth}
%
\parbox[t][\columnheight]{\textwidth}{
%
\devskipcol
%
\begin{block}{Hyperbolicity of a sequence}
\begin{minipage}{0.97\textwidth}
\begin{deffnon}[Pilyugin, 2006] We say that a sequence $\mathcal{A}$ is hyperbolic on $I$ if there exist constants $K > 0,$ $\lambda\in (0,1)$ and projections $P_k, Q_k,\ k\in I$ such that if $S_k=P_k\Rb^d$ and $U_k=Q_k\Rb^d$ then the following holds:
\begin{IEEEeqnarray*}{c}
\Rb^d=S_k \oplus  U_k; \\
A_kS_k=S_{k+1},\ A_kU_k=U_{k+1};\\
|\Phi_{k,l}v|\leq K\lambda^{k-l}|v|,\ v\in S_l,\ k\geq l; \label{hypdef:stable1}\\
|\Phi_{k,l}v|\leq K\lambda^{l-k}|v|,\ v\in U_l,\ k\leq l; \label{hypdef:unstable1}\\
\normop{P_k},\normop{Q_k} \leq K. \label{hypdef:projbound}
\end{IEEEeqnarray*}
Everywhere here we mean that all indices are from $I.$
\end{deffnon}
\end{minipage}
\end{block}
%
%
\begin{block}{Theorems of Maizel and Pliss}
\begin{minipage}{0.97\textwidth}
\begin{theoremnon} [a generalization of the discrete analog of Maizel Theorem] \label{thm:Maizelcor}
Let $I=\Zplus$ and the norms of all matrices $A_k$ and $\inv{A_k}$ be bounded by $M>0.$ A sequence $\Ac$ has property $B_\omega(I)$ iff it is hyperbolic on  $\Zplus.$
\end{theoremnon}

\begin{theoremnon} [a generalization of the discrete analog of Pliss Theorem] \label{thm:mainpliscor}
Let $I=\Zb$ and the norms of all matrices $A_k$ and $\inv{A_k}$ be bounded by $M>0.$ A sequence $\Ac$ has property $B_\omega(I)$ iff it is hyperbolic on both  $\Zplus$ and $\Zminus$ and the spaces $B^+(\Ac)$ and $B^-(\Ac)$ are transverse. Here
\begin{eqnarray*}
B^+(\Ac)=\setdef{v\in\Rb^d}{\vmod{\Phi_{k,0}v}\to 0,\ k\to +\infty}, \\
B^-(\Ac)=\setdef{v\in\Rb^d}{\vmod{\Phi_{k,0}v}\to 0,\ k\to -\infty}.
\end{eqnarray*}
\end{theoremnon}

There are lots of papers concerning results of this type. They generalize theorems of Maizel and Pliss in different directions (Latushkin, Baskakov, Sasu). For infinite dimensional equations and/or different classes of spaces having certain homogeneity properties. Spaces $\Nc_\omega(I)$ are neither homogeneous in the sense of Baskakov nor translation-Invariant in the sense of Sasu. 
\input{6_discrthms2.tex}
\end{minipage}
\end{block}
%
%
\begin{block}{New sort of shadowing}
\begin{minipage}{0.97\textwidth}
\input{7_decrshad.tex}
\end{minipage}
\end{block}
%
}

\end{column}

\end{columns}


\end{frame}
\end{document}


%%% Local Variables:
%%% mode: latex
%%% TeX-master: t
%%% TeX-PDF-mode: t
%%% End:
