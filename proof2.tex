There is a relation between the solvability of the system of linear difference equations and some hyperbolicity properties of this system.
\newline
Let $I$ be either $\Zplus = \setdef{k\in\Zb}{k\geq 0}$ or $\Zminus = \setdef{k\in\Zb}{k\leq 0}$ or $\Zb.$

Let $\Ac=\{A_k\}_{k\in I}$ be a sequence of linear isomorphisms $\Rb^d\to\Rb^d$ indexed by integers from $I.$ Consider homogeneous and inhomogeneous equations associated with this sequence.

\begin{IEEEeqnarray*}{rCll}
x_{k+1}&=&A_k x_k,\qquad &k\in I; \label{eq:homogen} \\
x_{k+1}&=&A_k x_k + f_{k+1},\qquad &k\in I. \label{eq:nonhomogen}
\end{IEEEeqnarray*}

Fix $\omega \geq 0.$ Denote the Banach space of sequences with bounded norm $\normban{x}_\omega=\sup\limits_{k\in I} \vmod{x_k} (|k|+1)^\omega$ by $\Nc_{\omega}(I).$ Such spaces have been already studied in a similar context (Bichekguev, 2011).

\begin{deffnon}
A sequence $\Ac$ has \indef{Perron property $B_{\omega}(I)$ } if $\seqz{f}{k}\in \Nc_\omega(I)$ \imply solution of inhomogeneous equation $\seqz{x}{k} \in \Nc_\omega(I).$
\end{deffnon}

Such properties are often called \indef{admissibility.}


