\section{Actions of nilpotent groups.}
We start from the proof of Lemma \ref{lemNormSubgroup}.

\begin{proof}[Proof of Lemma \ref{lemNormSubgroup}]
%%%%%%%%%%%%%%%%%%%%%%%%%%%%%%%%%

Fix a finite symmetric generating set $S_H$ in $H$ and continue it to a finite symmetric generating set $S$ in $G$. By Proposition \ref{prop}, we can assume that our initial generating set $S$ was chosen in this way.

Let $\Delta, \gamma > 0$ be the constants from the definitions of a topologically Anosov action and expansivity for $\Phi|_H$.
Since the maps $\{f_s\}_{s \in S}$ are uniformly continuous, there exists $\delta < \min(\Delta/3, \gamma)$ such that
\begin{equation}\label{eqUnCont}
\dist(f_s(\omega_1), f_s(\omega_2)) < \Delta/3
\end{equation}
for any $s \in S$ and any two points $\omega_1, \omega_2 \in \Omega$ satisfying $\dist(\omega_1, \omega_2) < \delta$.

Fix $\ep \in (0, \delta)$ and choose $d < \ep$ from the definition of shadowing for $\Phi|_H$ for the generating set $S_H$. Fix a $d$-pseudotrajectory $\{y_g \in V\}_{g\in G}$ of~$\Phi$.

For any element $q\in G$ consider the sequence $\{z_h = y_{hq}\}_{h\in H}$. Note that this sequence is a $d$-pseudotrajectory of $\Phi|_{H}$. Since $\Phi|_{H}$ is topologically Anosov with respect to $(U, V)$, there exists a unique point $x_q\in V$ such that
\begin{equation}
\label{shad1}
\dist(z_h,\Phi(h,x_q))=\dist(y_{hq},f_{h}(x_q)) < \ep, \quad h \in H.
\end{equation}
Existence of such $x_q$ follows from (TA2), uniqueness follows from (TA1), (TA3), and the inequality $\ep < \gamma$.

Let us prove that $\{x_q\}_{q\in G}$ is an exact trajectory.

Fix $s \in S$ and $q \in G$. Consider an arbitrary element $h\in H$. Since $H$ is a normal subgroup of $G$, there exists an element $h'\in H$ such that
\begin{equation}
\label{connect}
sh'=hs.
\end{equation}

It follows from \sref{eqUnCont}--\sref{connect} that
\begin{equation}
\label{ineq1}
\mbox{dist}(y_{sh'q},f_{h}(x_{sq})) < \epsilon,
\end{equation}
\begin{equation}
\label{shad2}
\mbox{dist}(f_s(y_{h'q}),f_s(f_{h'}(x_q))) < \Delta/3.
\end{equation}

Since $\{y_g\}_{g\in G}$ is a $d$-pseudotrajectory for $\Phi$, it follows from \sref{connect}--\sref{shad2} that
\begin{multline*}
\dist(f_{h}(x_{sq}),f_{h}(f_s(x_q)))\leq \\ \dist(f_{h}(x_{sq}), y_{sh'q}) + \dist(y_{sh'q}, f_{s}(y_{h'q}))+
\dist(f_{s}(y_{h'q}), f_{hs}(x_q)))\leq \\ \epsilon + d + \Delta/3 < \Delta.
\end{multline*}

Due to expansivity of $\Phi|_{H}$ on $U$, we conclude that
\begin{equation}\notag
\label{goal}
x_{sq} = f_s(x_q), \quad s \in S, q \in G.
\end{equation}
Since $S$ is a generating set for $G$, these equalities imply that
$x_q = f_q(x_e)$, for  all $q \in G$, and hence by~\sref{shad1}  $x_e$ satisfies inequalities \sref{eqDefSh}.

Expansivity of $\Phi$ is trivial.
\end{proof}
%\end{equation}



%Let us prove that
%\begin{equation}\label{eqTraj}
%_q = f_q(x_e), \quad q \in G.
%\end{equation}
%Note that in that case inequalities \sref{shad1} imply that $x_e$ is desired %element for the shadowing property. Clearly in order to prove \sref{eqTraj} %it is enough to prove
%\begin{equation}
%label{goal}
%y_{sq} = f_s(y_q), \quad s \in S, q \in G.
%\end{equation}



%%%%%%%%%%%%%%%%%%%%%%%%%%%%%%%%%%%%%%%


Next we prove Theorem \ref{thmVirtNilp} for the case of nilpotent groups.
\begin{lem}\label{lemNilp}
Let $G$ be a finitely generated nilpotent group of class $n$ and $\Phi$ be an uniformly continuous action of $G$ on a metric space $\Omega$. Assume that there exists an element $g \in G$ such that $f_g$ is topologically Anosov with respect to $(U, V)$.
Then the action $\Phi$ is topologically Anosov with respect to $(U, V)$.
\end{lem}
\begin{proof}
Let us prove this lemma by induction on $n$.

For $n = 0$ the group $G$ is abelian and hence the group $P = \left<g\right>$ generated by $g$ is a normal subgroup of $G$. Since $f_g$ is topologically Anosov, applying Lemma~\ref{lemNormSubgroup} we conclude that $\Phi$ has the shadowing property.

Let $n \geq 1$ and assume that we have proved the lemma for all nilpotent groups of class less or equal $n-1$. Denote $Q = [G, G]$ and $P = \left< Q, g \right>$ (i.e. $P$ is the minimal subgroup of $G$ that contains $Q$ and $g$). 
\begin{prp}
\label{nilpprop}
%Let $G$ be a nilpotent group of class $n$, put $Q=[G,G]$. Fix $g\in G$ and %put $P=<Q,g>$.
\begin{itemize}
  \item[(N1)] The group $P$ is a normal subgroup of $G$.
    \item[(N2)] The group $P$ is nilpotent of class at most $n-1$.
  %\item[(N3)] The group $P$ is finitely generated.
\end{itemize}
\end{prp}

\begin{proof}[Proof of Proposition \ref{nilpprop}.]
Let us start from Item (N1). Fix arbitrary $p\in P$, $h\in G$. Note that $hph^{-1}\in [G,G]p=Qp\subset P$, which proves the claim.

Let us prove Item (N2).
It is clear that any subgroup of a nilpotent group of class $n$ is a nilpotent group of class at most $n$. However we need a stronger result for the subgroup $P$. As the analysis of simple examples shows (e.g. the direct product of the Heisenberg group and $\mathbb{Z}$), a nilpotent group of class $n$ may have proper subgroups of class $n$. So item (N2) is not trivial.

Denote
\begin{equation}\notag
\label{qdef}
R = [Q,G]=[[G,G],G].
\end{equation}
Clearly, in order to prove (N2) it is sufficient to prove that
\begin{equation}\notag
\label{tempgoal}
[P,P]\subset [[G,G],G] = R
\end{equation}
(since it implies $[[P,P],P]\subset [[[G,G],G],G]$ and etc.).

Since $Qg=gQ$, any element $p\in P$ has a representation as $qg^k$ for some $q\in Q$, $k\in\mathbb{Z}$. Fix $p_1,p_2\in P$ and put $p_1=q_1g^{k_1}, p_2=q_2g^{k_2}$.
%By $(\ref{qdef})$ there exists $r_1\in R$ such that
Note that
$$
p_1p_2=q_1g^{k_1} q_2g^{k_2} = q_1r_1q_2g^{k_1+k_2} = r_2q_1q_2g^{k_1+k_2},
$$
\begin{equation}\notag
%\label{mform2}
p_2p_1 = q_2g^{k_2}q_1g^{k_1} = q_2r_3q_1g^{k_1 + k_2} = r_4q_2q_1g^{k_1+k_2} = r_5q_1q_2g^{k_1+k_2}
\end{equation}
for some $r_1, \dots, r_5 \in R$, and hence $[p_1,p_2] = r_2r_5^{-1}\in R.$


%Note that $q_1r_1q_1^{-1}r^{-1}\in [Q,G] = R$, hence there exists $r_2\in R$  %such that
%\begin{equation}
%\label{mform1}
%p_1p_2 = r_2q_1q_2g^{k_1+k_2}.
%\end{equation}
%In an analagous way there exist $r_3, r_4\in R$ such that
%It follows from $(\ref{mform1})$ and $(\ref{mform2})$ that
%$[p_1,p_2] = r_2r_4^{-1}\in R.$

%Note that Item (N3) holds not only for nilpotent groups, but also for a more general class of polycyclic groups (cf. Definition \ref{polycdef}), since any subgroup of a polycyclic group is finitely generated.

\end{proof}

Let us note that these properties strongly use nilpotency of $Q$, and their analogs do not hold, for example, for solvable groups.

Let us continue the proof of Lemma \ref{lemNilp}. Since $P$ is a finitely generated (due to Remark \ref{remFinGen}) nilpotent group of class at most $n-1$, $g \in P$, and $f_g$ is topologically Anosov, by the induction assumptions we conclude that $\Phi|_P$ is topologically Anosov. Combining this property, (N1) and Lemma \ref{lemNormSubgroup} we conclude that $\Phi$ has the shadowing property.
\end{proof}

\begin{proof}[Proof of Theorem \ref{thmVirtNilp}]
Since $G$ is virtually nilpotent, there exists a nilpotent normal subgroup $H$ of $G$ of finite index. Due to Remark \ref{remFinGen} the group $H$ is finitely generated. Consider $g\in G$ from the assumptions of the theorem. Since $H$ is a subgroup of finite index, there exists $k > 0$ such that $g^k \in H$. Since $f_g$ is topologically Anosov, the map $f_g^k = f_{g^k}$ is also topologically Anosov. Hence, by Lemma \ref{lemNilp}, the action $\Phi|_{H}$ is topologically Anosov. Applying Lemma \ref{lemNormSubgroup} we conclude that $\Phi$ is topologically Anosov too.
\end{proof}


