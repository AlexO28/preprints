\section{An action of a Baumslag-Solitar group.}

\begin{proof}[Proof of Theorem \ref{ThBS}]

Without loss of generality, by Proposition~\ref{prop}, we consider the group $BS(1,n)=<a,b\mid aba^{-1}=b^n>$ with the standard generating set $S=\{a,b,a^{-1},b^{-1}\}$.

\textbf{Proof of Item (i).} To derive a contradiction assume that $\Phi$ has shadowing and choose $d > 0$ from the definition of the shadowing property applied to $\epsilon=1$.

Consider an auxiliary action $\Psi:  G \times (\RR \times \ZZ) \to (\RR \times \ZZ)$ generated by the maps
$$
g_a(x, k) = (x + n^{-k}, k), \quad g_b(x, k) = (x, k+1).
$$
It is easy to check that $g_b \circ g_a = g_a^n \circ g_b$, and hence the action $\Psi$ is well defined.


Consider the map $F : (\RR \times \ZZ) \to \RR$ defined as follows
$$
F(x, k) =
  \begin{cases}
  0, & \quad k <0, \quad \mbox{or} \quad x \leq 0,\\
  x \lambda^k, & \quad x \in (0, 1), \; k \geq 0,\\
  \lambda^k, & \quad x \geq 1, \; k \geq 0.\\
  \end{cases}.
$$

Finally, consider the sequence
$$
y_g = F(\Psi(g, (0, 0))) \cdot d/2, \quad g \in G.
$$
We claim that $\{y_g\}_{g\in G}$ is a $d$-pseudotrajectory for the action $\Phi$, i.e. inequalities $(\ref{dpstdef})$ hold for $s\in \{a,b,a^{-1},b^{-1}\}$.

Indeed, fix $g \in G$. Denote $(x, k) = \Psi(g, (0, 0))$.

\textit{Case $s = a$} (case $s=a^{-1}$ is treated similarly). Note that $$\Psi(ag, (0, 0)) = (x + n^{-k}, k),$$
$$
|y_{ag} - f_a(y_g)| = |F(x + n^{-k}, k) - F(x, k)|\cdot d/2.
$$

If $k < 0$ or $x \in (-\infty, -n^{-k}] \cup (1, +\infty)$, then
$
F(x + n^{-k}, k) = F(x, k).
$

Otherwise $k \geq 0$ and one of the following holds:
\begin{itemize}
\item $x \in (-n^{-k}, 0]$ and $|F(x + n^{-k}, k) - F(x, k)| = |(x+n^{-k})\lambda^k|\leq n^{-k}\lambda^k \leq 1;$
\item $x \in (0, 1- n^{-k}]$ and
$|F(x + n^{-k}, k) - F(x, k)| = n^{-k}\lambda^{k}\leq 1;$
\item $x \in (1- n^{-k}, 1]$ and $|F(x + n^{-k}, k) - F(x, k)| = \lambda^k (1-x)\leq \lambda^k n^{-k}\leq 1.$
\end{itemize}

Thus in all cases
$|y_{ag} - f_a(y_g)| < d$, which proves (\ref{dpstdef}).

\textit{Case $s = b$} (case of $s=b^{-1}$ is treated similarly). Note that
$$\Psi(bg, (0, 0)) = (x, k+1),$$
$$
|y_{bg} - f_b(y_g)| = |F(x, k) - \lam F(x, k-1)|\cdot d/2.
$$

If $k \ne 0$ or $x \notin [0, 1]$, then $F(x, k) = \lam F(x, k-1)$.

Otherwise $k = 0$, $0\leq x\leq 1$, and hence
$|F(x,k) - \lambda F(x,k-1)| = |x|\leq1.$


Thus $|y_{bg} - f_b(y_g)| < d$, which proves (\ref{dpstdef}).

Since by our assumptions the action $\Phi$ has the shadowing property, there exists $x_e \in \RR$ such that
$(\ref{eqDefSh})$ holds.

Note that $y_{b^k} = 0$ for $k \geq 0$. Substituting $g = b^k$ into (\ref{eqDefSh}), we conclude that
$$
|\lam^k x_e| < 1, \quad k \geq 0
$$
and hence $x_e = 0$.

Now substitute $g = b^k a$ into $(\ref{eqDefSh})$:
$$
|\lambda^kd/2 - 0|<1, \quad k  \geq 0,
$$
which is impossible for sufficiently large $k$.
The derived contradiction finishes the proof of Item (i).

\medskip

\textbf{Proof of Item (ii).} Fix $\ep > 0$. Note that the map $f_b$ has shadowing and expansivity. Let us choose $d \in (0, \ep)$ such that any $d$-pseudotrajectory of $f_b$ can be $\ep$-shadowed by an exact trajectory of $f_b$. Consider an arbitrary $d$-pseudotrajectory $\{y_g\}_{g \in G}$ of the action $\Phi$.

For any element $q\in G$ consider the sequence $\{z_k\}_{k\in \ZZ}$, defined by $z_k= y_{b^k q}$. Note that this sequence is a $d$-pseudotrajectory for $f_b$. Since $f_b$ has shadowing and expansivity, there exists a unique point $x_q\in \RR$ such that
\begin{equation}
\label{BSshad1}
|z_k - f_b^k(x_q)|=|y_{b^k q}-f_{b}^k(x_q)| < \ep, \quad k \in \ZZ.
\end{equation}
We claim that $x_q = \Phi(q, x_e)$. To prove this, it is enough to show that
\begin{equation}\label{eqBSshad2}
x_{bq} = \lam x_q, \; x_{aq} = x_q, \quad q \in G.
\end{equation}
The first equality follows directly from expansivity of $f_b$. Let us prove the second one.

Note that the relation $ba = a^n b$ implies that \begin{equation}\label{eqGroupRel}
b^ka = a^{(n^k)}b^k, \quad k > 0.
\end{equation}
Fix an arbitrary $q \in G$. Relations (\ref{BSshad1}), (\ref{eqGroupRel}), and the definition of a pseudotrajectory imply that for any $k > 0$ the following relations hold:
$$
|\lam^k x_{aq} - y_{b^kaq}| < \ep,
$$
$$
y_{b^kaq} = y_{a^{(n^k)}b^kq},
$$
$$
|y_{a^{(n^k)}b^kq} - y_{b^kq}| < d n^k,
$$
$$
|y_{b^kq} - \lam^k x_q| < \ep,
$$
and hence
$$
|\lam^k(x_{aq} - x_q)| < 2\ep + d n^k, \quad\forall k > 0.
$$
Since $\lam > n$, it proves the second equality in~\sref{eqBSshad2} and finishes the proof of Item (ii).

\end{proof}
