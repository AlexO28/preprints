%\documentclass[final,notheorems,hyperref={pdfpagelabels=false}]{beamer}
\documentclass[final,notheorems]{beamer}
%\documentclass{beamer}

\mode<presentation> {
  \usetheme{Tau}
}

%\usepackage[orientation=portrait,size=a0,scale=1.4,debug]{beamerposter}
\usepackage[orientation=portrait,size=a0,scale=1.4,debug]{beamerposter}
%\usepackage [russian]{babel}

%\usefonttheme{serif}
\usefonttheme{professionalfonts}

%%%mine
%\usepackage[x11names,rgb]{xcolor}
\usepackage{graphicx}
\usepackage{color}
\usepackage{transparent}

\usepackage[english]{babel}
\usepackage{multirow}
\usepackage{amsmath}
\usepackage{amsfonts,amssymb}
\usepackage[retainorgcmds]{IEEEtrantools}
\usepackage{verbatim}

\usepackage{setspace}

\definecolor{defwordcolor}{rgb}{0.7,0.0,0.0}
\setbeamercolor{alerted text}{fg=defwordcolor}

%%%%%%%%%%%%%%%%%%%%%%%%%%%%%%%%%%%%%%%
%%%%%%%%%%%%%%%%%%%%%%%%%%%%%%%%%%%%%%%
%%%%%%%%%%%%%%%%%%%%%%%%%%%%%%%%%%%%%%%
%%%%%%%%%%%%%%%%%%%%%%%%%%%%%%%%%%%%%%%

%\texttt

%\newcommand{\devskip}{\vskip-60pt}
%\newcommand{\devskipcol}{\vskip-35pt}
%\newcommand{\devskipcols}{\vskip-30pt}

\newcommand{\indef}[1]{\alert{{#1}}}

\newcommand{\devskip}{\vskip-60pt}
\newcommand{\devskipcol}{\vskip-30pt}
\newcommand{\devskipcols}{}

\theoremstyle{plain} \newtheorem{theorem}{Theorem}
\theoremstyle{plain} \newtheorem*{theoremnon}{Theorem}
\theoremstyle{definition} \newtheorem{deff}{Definition}
\theoremstyle{definition} \newtheorem*{deffnon}{Definition}
\theoremstyle{remark} \newtheorem{rem}{Remark}

\makeatletter
\setbeamertemplate{theorem begin}
{%
\vspace{0.5em}
\slshape %ignore body font
%\begin{\inserttheoremblockenv}
{%
\inserttheoremheadfont
\inserttheoremname
\inserttheoremnumber
\ifx\inserttheoremaddition\@empty
\else\ (\inserttheoremaddition)\fi%
\inserttheorempunctuation\hskip20pt
}%
}
\setbeamertemplate{theorem end}
{
%\end{\inserttheoremblockenv}
\vspace{0.5em}
}
\makeatletter


\institute[Chebyshev laboratory]{Chebyshev laboratory, Department of Mathematics and Mechanics, Saint Petersburg State University, Saint Petersburg, Russia}

\subject{MDS� 2012}

\title{\huge Nondensity of orbital shadowing in $C^1$-topology.}
\author[Alexey Osipov]{Alexey Osipov \\ osipovav28@googlemail.com}
%\mail{osipovav28@googlemail.com}
\mail{Supported by travel grant of RFFI, project 12-01-09302-mob\_z, and Chebyshev Lab.}


\newlength{\columnheight}
\setlength{\columnheight}{105cm}
%%%%%%%%%%%%%%
%%%%%%%%%%%%%%
\begin{document}

\begin{frame}
\devskipcols

%\begin{block}{Problem}
%
%\end{block}

\begin{block}{Problem}
\center{\LARGE{Is a set of dynamical systems having some shadowing property generic or nondense?}}
\end{block}

%\devskipcol
%\devskipcol

\begin{columns}[t]

% Col1
\begin{column}{.325\textwidth}
\parbox[t][\columnheight]{\textwidth}{
%
\devskipcol
%
\begin{block}{What is shadowing?}
\begin{minipage}{0.97\textwidth}
Theory of shadowing studies relations between approximate and exact trajectories of dynamical systems on unbounded time intervals. Shadowing means that any sufficiently precise approximate trajectory (a pseudotrajectory) is close to some exact trajectory.
\end{minipage}
\end{block}

\begin{block}{POTP}
\begin{minipage}{0.97\textwidth}
Let $f$ be a homeomorphism of a metric space $(M,\mbox{dist})$.
\begin{deff}
A sequence $\{x_k\}_{k\in\mathbb{Z}}$ of points of $M$ is a \indef{$d$-pseudotrajectory} of $f$ iff
$$\mbox{dist}(x_{k+1},f(x_k))<d,\quad\forall k\in\mathbb{Z}.$$
\end{deff}

\devskip

\begin{deff}
A diffeomorphism $f$ has \indef{pseudoorbit tracing property} or \indef{standard shadowing property} ($f\in\mbox{POTP}$) iff for any $\epsilon>0$ there exists a $d>0$ such that for any $d$-pseudotrajectory $\xi=\{x_k\}_{k\in\mathbb{Z}}$ there exists a point $p\in M$ such that
$$\mbox{dist}(x_{k},f^k(p))<\epsilon.$$
\end{deff}

%\input{fig1.eps_tex}
%\includegraphics{fig1.pdf_tex}
\devskip

\begin{figure}
\centering
\input{fig1.pdf_tex}	
\caption{Example when POTP holds: black denotes a pseudo\-tra\-jec\-tory, red denotes an exact trajectory.}
\end{figure}

\textit{Any structurally stable diffeomorphism has POTP} (\textbf{Robinson, Sawada, Morimoto, 1977--80}).

\end{minipage}
\end{block}

\begin{block}{OSP and WSP}
\begin{minipage}{0.97\textwidth}

By $N(\epsilon,A)$ denote the $\epsilon$-neighborhood of $A\subset M$. By $O(p,f)=\{f^k(p)\}_{k\in\mathbb{Z}}$ denote the trajectory of $p$.

\begin{deff}
A diffeomorphism $f$ has \indef{orbital shadowing property} ($f\in\mbox{OSP}$) iff for any $\epsilon>0$ there exists a $d>0$ such that for any $d$-pseudotrajectory $\xi=\{x_k\}_{k\in\mathbb{Z}}$ there exists a point $p\in M$ such that
$$\xi\subset N(\epsilon, O(p,f))\quad\mbox{and}\quad O(p,f)\subset N(\epsilon,\xi).$$
\end{deff}

\devskipcol

Example of $f\in \mbox{OSP}\backslash \mbox{POTP}$: \textit{irrational rotation of the circle}.

\begin{deff}
A diffeomorphism $f$ has \indef{weak shadowing property} ($f\in\mbox{WSP}$) iff for any $\epsilon>0$ there exists a $d>0$ such that for any $d$-pseudotrajectory $\xi=\{x_k\}_{k\in\mathbb{Z}}$ there exists a point $p\in M$ such that
$$\xi\subset N(\epsilon, O(p,f)).$$
\end{deff}

\devskipcol

Example of $f\in \mbox{WSP}\backslash \mbox{OSP}$: \textit{the Plamenevskaya map} (an $\Omega$-stable diffeomorphism constructed in 1999).
\end{minipage}
\end{block}

\begin{block}{Spaces of dynamical systems}
\begin{minipage}{0.97\textwidth}
Let $M$ be a closed smooth Riemannian manifold. By $H(M)$ denote the space of all homeomorphisms of $M$ with $C^0$-metric.
By $\mbox{Diff}(M)$ denote the set of all $C^1$-diffeomorphisms of $M$ with $C^1$-metric.
\begin{deff}
A set is called \indef{generic} if it is a Baire second cathegory set (i.e. contains a countable union of open and dense sets) in the corresponding space.
\end{deff}
\devskip
\end{minipage}
\end{block}

%\begin{block}{Acknowledgment}
%Supported by RFFI grant
%\end{block}

}
\end{column}


% Col2
\begin{column}{.325\textwidth}
\parbox[t][\columnheight]{\textwidth}{
%
\devskipcol
%
\begin{block}{Previous results}
\begin{minipage}{0.97\textwidth}
\begin{figure}
\centering
%\def\svgwidth{500pt}
\includegraphics{fig2.pdf}	
\caption{Previous results on the topic.}
\end{figure}
\end{minipage}
\end{block}

\begin{block}{Main result: $C^1$-nondensity of OSP}
\begin{minipage}{0.97\textwidth}
\begin{theoremnon}[Osipov,2010]
There exists a domain $W\subset\mbox{Diff}(S^2\times S^1)$ such that any $f\in W$ does not have OSP.
\end{theoremnon}

\textit{It is possible to construct similar domains in $\mbox{Diff}(M)$ for any $n$-dimensional manifold $M$ with $n\geq 3$.}
\end{minipage}
\end{block}

\begin{block}{General strategy of the proof}
\begin{minipage}{0.97\textwidth}
In essense, we use a strategy of Ilyashenko and Gorodetski developed for construction of domains in $\mbox{Diff}(M)$ with specific properties.

\textbf{Step 1.} Prove that there exists a \textit{sufficiently good} neighborhood $\mathcal{U}$ in the space of skew products of a certain type such that any skew product from $\mathcal{U}$ does not have OSP.

\textbf{Step 2.} For transfer from skew products to diffeomorphisms apply the result of Gorodetski that can be informally formulated as follows:

\begin{theoremnon}[Gorodetski, 2001]
Let $\mathcal{U}$ be a \textit{sufficiently good} neighborhood in the space of skew products of a certain type. Then there exists a $C^1$-domain $\mathcal{V}\subset\mbox{Diff}(M)$ such that any $f\in \mathcal{V}$ has a partially hyperbolic locally maximal invariant set $\Delta$, and $F|_{\Delta}$ is topologically conjugate with a skew product from $\mathcal{U}$. 
\end{theoremnon}

\devskip
\devskipcol

\end{minipage}
\end{block}

\begin{block}{Skew products}
\begin{minipage}{0.97\textwidth}
Note that skew products can be interpreted as random dynamical systems.

Let $\Sigma^2$ be a space of biinfinite sequences of 0 and 1. Let $\sigma:\Sigma^2\mapsto\Sigma^2$ be a Bernoulli shift given~by
$$(\sigma(\omega))_{k} = \omega_{k+1},\quad\forall\omega\in\Sigma^2, k\in\mathbb{Z}.$$
\begin{deff}
Let $\{f_{\omega}\}_{\omega\in\Sigma^2}$ be a family of diffeomorphisms of $S^1$. A map $f:\Sigma^2\times S^1\mapsto\Sigma^2\times S^1$ is called a \indef{mild skew product} iff 
$$f(\omega,\phi)=(\sigma(\omega),f_{\omega}(\phi)).$$ 
\end{deff}

\devskip

Let $g_0$ be a rotation of $S^1$ by angle $b$, $g_1$ be a Morse-Smale diffeomorphism with two hyperbolic fixed points $p_1$ and $p_2$, and $Df(p_2)=a<1$, $Df(p_1)=1/a$ (cf. the figure).

\begin{figure}
\centering
\label{fig3}
\input{fig3.pdf_tex}	
\caption{Diffeomorphisms $g_0$ and $g_1$.}
\end{figure}

%Consider a mild skew product $G_{a,b}$ generated by
%$f_{\omega}:=f_{\omega_0}$ for $\omega\in\Sigma^2$.
Let $a$ be close to 1, $b$ be close to 0, and $\delta$ be a sufficiently small number;
then $\mathcal{U}$ is a set of all mild skew products $\{f_{\omega}\}_{\omega\in\Sigma^2}$ such that 
\begin{itemize}
	\item $f_{\omega}$ is $\delta$-close to $g_0$ in $\mbox{Diff}(S^1)$ if $\omega_0=0$;
  \item $f_{\omega}$ is $\delta$-close to $g_1$ in $\mbox{Diff}(S^1)$ if $\omega_0=1$.
\end{itemize}

\devskip

\end{minipage}
\end{block}
}
\end{column}


% Col3
\begin{column}{.325\textwidth}
\parbox[t][\columnheight]{\textwidth}{
%
\devskipcol

\begin{block}{Case 1}
\begin{minipage}{0.97\textwidth}
%Let $G$ be a mild skew product from $\mathcal{U}$. Assume that there exist two hyperbolic periodic points $q_1$ and $q_2$ such that $W^u(q_1)\cap W^s(q_2)\neq\emptyset$.
Let $F$ be a diffeomorphism from $\mathcal{V}\subset\mbox{Diff}(S^2\times S^1)$ which is assigned to some mild skew product $G$. 

Assume that there exist two hyperbolic periodic points $q_1$ and $q_2$ lying on different fibres such that $$W^u(q_1)\cap W^s(q_2)\neq\emptyset.$$

\begin{figure}
\centering
\label{fig4}
\input{fig4.pdf_tex}	
\caption{$F$ in Case 1: red denotes a pseudotrajectory, blue denotes fibres.}
\end{figure}

%It is possible to construct a pseudotrajectory that can not be orbitally close to any exact trajectory (cf. the figure).
Using expansivity of the Bernoulli shift, we prove that the constructed pseudotrajectory (cf. the figure) can not be orbitally close to any exact trajectory.

\textit{Note that in order to derive contradiction it is important to prove that the intermediate part of the pseudotrajectory is far from the points $q_1$ and $q_2$.}

\end{minipage}
\end{block}

\begin{block}{Case 2}
\begin{minipage}{0.97\textwidth}
Let $F$ be a diffeomorphism from $\mathcal{V}\subset\mbox{Diff}(S^2\times S^1)$ which is assigned to some mild skew product $G$. 

Any such $F$ has the following property:

\textit{hyperbolic periodic points with different indices are dense in the partially hyperbolic invariant set $\Delta$.}

Since Case 1 does not hold, for any two hyperbolic periodic points $q_1$ and $q_2$ lying on different fibres $$W^u(q_1)\cap W^s(q_2)=\emptyset.$$

\begin{figure}
\centering
\label{fig5}
\input{fig5.pdf_tex}	
\caption{$F$ in Case 2: red denotes a pseudotrajectory, blue denotes fibres.}
\end{figure}

Using density of hyperbolic periodic points with different indices, we contruct a pseudotrajectory joining $r_1$ and $r_4$ (cf. the figure). Assume that this pseudotrajectory is orbitally close to an exact trajectory of some point $p$.

Using methods from theory of skew products and symbolic dynamics, we prove that 
$$p\in W^u(r_1)\cap W^u(r_4)$$
which contradicts to our conditions.

\textit{Note that in order to derive contradiction it is important to prove that the intermediate part of the pseudotrajectory is far from the points $r_1$ and $r_4$. In fact it is the key technical difficulty.}

\end{minipage}
\end{block}

\begin{block}{Possible directions for further research.}
\begin{minipage}{0.97\textwidth}

\begin{enumerate}
	\item \textbf{Conjecture:} OSP is $C^1$-generic for two-dimen\-sio\-nal manifolds.
	\item One-sided shadowing properties.
		\item Inverse shadowing properties.
				\item Limit shadowing.
\end{enumerate}

\end{minipage}
\end{block}

}
\end{column}

\end{columns}

%\begin{block}{HRUSHA}
%Hrusha!
%\end{block}

\end{frame}
\end{document}


%%% Local Variables:
%%% mode: latex
%%% TeX-master: t
%%% TeX-PDF-mode: t
%%% End:
