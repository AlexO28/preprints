\section{Introduction. }

Theory of shadowing is now a sufficiently well-developed branch of theory of dynamical systems (see monographs \cite{PilBook, PalmBook} and a review of modern results~\cite{Pil2}). A dynamical system has a shadowing property if any sufficiently precise approximate trajectory (pseudotrajectory) is close to some exact trajectory. Shadowing theory plays an important role in theory of structural stability. The shadowing lemma \cite{Anosov, Bowen} is one of key results in theory of shadowing. It says that a dynamical system has the shadowing property in a small neighborhood of a hyperbolic set.

In parallel with a classical theory of dynamical systems (which studies actions of $\mathbb{Z}$ and $\mathbb{R}$), global qualitative properties of actions of more complicated groups were studied
(see book \cite{Katok2} and review \cite{Fisher}).
Paper \cite{PilTikh} introduced the shadowing property for actions of abelian groups $\mathbb{Z}^n \times \mathbb{R}^m$ for nonnegative integer $n$ and $m$.

In the present paper we introduce and study the shadowing property for actions of finitely generated, not necessarily, abelian groups.

For the case of finitely generated nilpotent groups we prove that an action of the whole group has shadowing (and expansivity) if the action of at least one element has shadowing and expansivity (Theorem~\ref{thmVirtNilp}).
This result can be viewed as a shadowing lemma for actions of nilpotent groups, since it implies that if an action of one element is hyperbolic, then the group action has the shadowing property. Note that in some cases an action of a  group is called hyperbolic if there exists an element which action is hyperbolic (see \cite{Barbot, Katok2, Hurder}). 

We show that our result cannot be directly generalized to the case of solvable groups. We consider a particular linear action of a solvable Baumslag-Solitar group (Theorem \ref{ThBS}) and demonstrate that the shadowing property has a more complicated nature, in particular, it depends on quantitative characteristics of hyperbolicity of the action.

We also consider actions of ''big groups'' (free groups, groups with infinitely many ends). In particular we show that there is no linear action of a non-abelian free group that has shadowing. This statement leads us to a question: which groups admit an action satisfying shadowing property?

These three results illustrate that the shadowing property depends not only on hyperbolic properties of actions of its elements but on the group structure as well.

The article is organized as follows. In Section 2 we give a definition of shadowing for actions of finitely generated groups. In Section 3 we recall necessary notions from group theory. In Section 4 we give precise statements of main results. Sections 5--7 are devoted to detailed consideration of actions of nilpotent, solvable Baumslag-Solitar, and free groups respectively.  In the appendix for consistency we prove Proposition~\ref{prop} about independence of shadowing on a choice of a generating set.

% Section~5 is devoted %to actions of nilpotent groups. In Section 6 we study the case of a solvable %Baumslag-Solitar group. In Section~7 we study actions of free groups.
